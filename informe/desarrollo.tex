\section{Desarrollo}

\subsection{Calibración}

El primer paso para comenzar el algoritmo de fotogrametría estéreo es realizar
la calibración del sistema. Cada imágen tiene una fuente de iluminación. Cada
fuente tiene una dirección en la que apunta su luz. El objetivo del paso de
calibración es poder calcular la dirección de cada fuene de iluminación.

Cada objeto tiene 12 imágenes del mismo con fuentes de iluminación con
distintas direcciones. Estas 12 fuentes de iluminación son distintas entre sí,
pero se repiten para todos los objetos. Esto hace posible que si pudiesen
calcularse las direcciones de las 12 fuentes para un objeto en particular,
entonces podrían reutilizarse para el resto de los objetos.

En este caso pueden calcularse las direcciones de las fuentes de iluminación
de uno de los objetos: la esfera. Lo que permite que esto sea posible es el
hecho de conocer con exactitud la profundidad del objeto. Al saber que el
objeto es una esfera, puede conocerse con exactitud la posición de cada
punto, sabiendo el centro y el radio de la misma siguiendo la siguiente
ecuación:

\begin{center}
$(x - x_0)^2 + (y - y_0)^2 + (z - z_0)^2 = r^2$
\end{center}

donde $z_0 = 0$ ya que suponemos que está centrada respecto del eje z.
Gracias a esto, puede calcularse el vector normal a cada punto de la esfera

\begin{center}
$n = (x - x_0, y - y0, z)$
\end{center}

Sabemos, dada la naturaleza de la esfera, que el punto de mayor intensidad
lumínica será el más cercano a la fuente de iluminación, por lo que la normal
de dicho punto apuntará en dirección a la fuente. El vector opuesto a esta
normal tiene la dirección de la fuente.

Si se ejecuta este procedimiento para las 12 imágenes de la esfera, entonces
el resultado será conocer la dirección de las 12 fuentes de iluminación. Esto
permite, en los pasos siguientes del algoritmo de fotogrametría estéreo,
poder calcular la normal para cada punto, basándose en que la fuente de
iluminacion $s_i$ con $1 \leq i \leq 12$ es la misma para la imágen $i$ de
cada objeto.
