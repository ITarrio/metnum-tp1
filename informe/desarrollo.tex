\section{Desarrollo}

\subsection{Calibración}

El primer paso para comenzar el algoritmo de fotometría estéreo es realizar
la calibración del sistema. Para cada objeto, contamos con 12 imágenes,
cada una con una fuente de iluminación distinta; es decir, la dirección o
ángulo de incidencia de la luz es diferente. El objetivo del paso de calibración
es calcular la dirección de cada fuente de iluminación.

A este efecto, contamos con un conjunto de imágenes de referencia iluminadas
con los mismos ángulos que el objeto a analizar, pero con la ventaja de que se
trata de una esfera. Al tener una forma conocida y simple, podemos conocer con
exactitud la profundidad y posición espacial de cada punto del objeto,
sabiendo el centro y el radio de la misma siguiendo la siguiente ecuación:

\begin{center}
$(x - x_0)^2 + (y - y_0)^2 + (z - z_0)^2 = r^2$
\end{center}

Si definimos $z_0 = 0$ (es decir, si asumimos que la esfera está centrada
respecto del eje z), podemos calcular el vector normal a cada punto de la esfera:

\begin{center}
$n = (x - x_0, y - y0, z)$
\end{center}

Para los propósitos de este trabajo, vamos a asumir que los objetos tienen
superficies lambertiana, por lo que la intensidad del brillo de un punto
particular solo depende de la orientación de la superficie respecto de la
fuente de iluminación. Es decir, el punto de mayor intensidad lumínica
será el más cercano a la fuente de iluminación, por lo que la normal
de dicho punto apuntará en dirección a la fuente. El vector opuesto a esta
normal tiene la dirección de la fuente.

Si se ejecuta este procedimiento para las 12 imágenes de la esfera, entonces
el resultado será conocer la dirección de las 12 fuentes de iluminación. Esto,
combinado con las imágenes correspondientes de la figura a procesar,
nos permitirá calcular la normal para cada punto de la figura.

\subsection{Reconstrucción del modelo 3D de los objetos digitalizados}

Una vez completada la calibración, ya se cuenta con las direcciones de
iluminación de cada caso, las cuales junto con la secuencia de imágenes de un
objeto nos proporcionarán la información necesaria para poder calcular todos
los puntos $(x,y,z)$ del modelo digitalizado y las normales de cada uno.

\subsubsection{Contrucción del campo normal}

En este paso se desea obtener la normal de cada punto del objeto. Para ello
se recurre a la ecuación 5 del enunciado de este trabajo práctico, que
vincula una matriz formada por 3 direcciones de fuentes de iluminación con el
vector $m$ y las intensidades del píxel en cada imágen. Esta matriz tiene 3
direcciones porque es la cantidad mínima de imágenes cuya fuente de
iluminación difiera entre sí necesaria para inferir la orientación de la
superficie.

La matriz de direcciones contiene la dirección de cada fuente de iluminación
dispuesta como fila. Una vez resuelto el sistema de ecuaciones se podrá
conocer $m$ y asi mediante la siguiente ecuacion $||m|| = |I_0\rho| ||n||$,
conocer el valor absoluto de $I_0\rho$. Esto permite conocer $n$ ya que tiene
la misma dirección que $m$ con la diferencia de que $n$ está normalizado
por ser la norma. Por lo que nos queda que $n = \frac{m}{||m||}$.

\subsubsection{Estimacion de la profundidad}

El siguiente paso es calcular la profundidad de cada pixel de la imagen. Para ello se recurre a las ecuaciones 11 y 12 del enunciado en donde se vincula a la normal de cada pixel con su profundidad. Estas ecuaciones nos permiten formar un sistema de ecuaciones como el de la ecuación 13.

Es necesario notar que $M$, la matriz de este sistema, es una matriz esparsa. La manera en la que la está dispuesta la información es la siguiente:

$M$ tiene dimensión $2n \times n$. Esto es porque necesitamos representar 2 ecuaciones por cada pixel, por lo que cada columna de $M$ referirá a $1$ pixel de la imagen y cada fila a una ecuación. Dado que cada pixel utiliza en las ecuaciones a los pixels de la derecha y de abajo, cada columna tendrá a lo sumo 4 elementos (dos por sus propias ecuaciones, una vez por el pixel de su izquierda y otra por el de arriba). Para el caso de los pixels de los bordes derechos e inferiores, solamente tendrán 2 elementos, que se dan por los pixeles que estan en la anteúltima columna y anteúltima fila de la imagen correspondientemente. El último pixel, el que se encuentra en la esquina inferior derecha, no tendrá ningún elemento en su respectiva columna.


Nuestra implementación consiste en considerar a esta matriz como un tipo especial de matriz (la esparsa), para la cual no se tiene un arreglo bidimensional como sería el caso habitual sino un vector de hashmap, donde cada elemento del vector representa una columna. Cada hashmap contendrá entonces 4 elementos o menos. Esto permite que al querer multiplicar una columna por otra (como ocurre en el caso de querer calcular $M^tM$) no sea necesario recorrer toda la columna sino únicamente los elementos distintos de 0. Esta optimización es crucial para el desarrollo del algoritmo a nivel temporal. Además, por cuestiones implementativas, se decidió utilizar un diccionario que mapea cada pixel a una columna, para así poder ubicar rápidamente en $z$ (una vez resuelto el sistema) el valor correspondiente a cada pixel.

Luego resta resolver el sistema de la ecuación 15. Para ello aplicamos la factorización de Cholesky, que también resulta optimizada debido a tratarse de una matriz esparsa, ya que para el cálculo de la matriz triangular inferior es necesario recorrer lo calculado en las iteraciones previas, y al utilizar la implementación de matriz esparsa definida anteriormente se logra recorrer la matriz solamente en los elementos distintos de 0.


\subsection{Experimentación}

En este trabajo se realizaron una serie de experimentaciones para poder responder preguntas planteadas en el enunciado. Se explica el detalle de las respuestas obtenidas en las siguientes subsecciones, pero a modo de introducción de esa parte indicamos que no se obtuvieron respuestas para todas las respuestas, pero las pruebas sirvieron para acercarse a una idea de las mismas.

\subsubsection{Análisis cuantitativo}



\subsubsection{Análisis cualitativo}

\paragraph{Luces por calibración VS Luces de la cátedra}

\paragraph{Calibración y repercusiones}

\paragraph{Elección de las luces}

A lo largo del desarrollo, se observó que utilizando las mismas luces se obtenian diferentes calidades de modelos 3D para las distintas imágenes.